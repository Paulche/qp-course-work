\documentclass[a4paper,fontsize=12pt,toc=bib]{scrartcl}

\usepackage[utf8]{inputenc}         % entering native characters is much convenient
\usepackage[russian]{babel}        % originally this document will be in german, but for review I will try to do my best in english
\usepackage[dvipsnames]{xcolor}     % refer to some colors by name
\usepackage{graphicx}               % able to load more graphic formats

% make clickable hyper links in documents
\usepackage[pdfborder={0 0 0},colorlinks,urlcolor=NavyBlue,linkcolor=NavyBlue,citecolor=NavyBlue]{hyperref}

\usepackage[headsepline,plainheadsepline]{scrpage2}     % for header and footer layout
\usepackage{nag}                                        % for not using forbidden things referenced in l2tabu
\usepackage{blindtext}                                  % for inserting blind text
\usepackage{geometry}                                   % for removing margins on titlepage
\usepackage[style=authoryear,bibstyle=authoryear,citestyle=authoryear,backend=bibtex]{biblatex} % for references
\usepackage{dtklogos}           % for BibTeX logo
\usepackage{xspace}             % inserts missing space at the end of custom commands
\usepackage{subfig}             % for subfigures
\usepackage{tabularx}           % for adaptive column widths
\usepackage{amsmath,amssymb}    % for cool math stuff
\usepackage{microtype}          % for better kerning

\setlength{\parindent}{0pt} %       We don't want any paragraph indentation
\addbibresource{literatur.bib}      % database for references
\graphicspath{{./images/}}          % sets the default path for images

% Enforce coloring for the whole author-year style and adjust hyperlinks as well
\DeclareCiteCommand{\cite}{\usebibmacro{prenote}{\color{NavyBlue}[}}{\usebibmacro{citeindex}\printtext[bibhyperref]{\usebibmacro{cite}}}{\multicitedelim}{\usebibmacro{postnote}{\color{NavyBlue}]}}
\DeclareCiteCommand*{\cite}{\usebibmacro{prenote}}{\usebibmacro{citeindex}\printtext[bibhyperref]{\usebibmacro{citeyear}}}{\multicitedelim}{\usebibmacro{postnote}}
\DeclareCiteCommand{\parencite}[\mkbibparens]{\usebibmacro{prenote}}{\usebibmacro{citeindex}\printtext[bibhyperref]{\usebibmacro{cite}}}{\multicitedelim}{\usebibmacro{postnote}}
\DeclareCiteCommand*{\parencite}[\mkbibparens]{\usebibmacro{prenote}}{\usebibmacro{citeindex}\printtext[bibhyperref]{\usebibmacro{citeyear}}}{\multicitedelim}{\usebibmacro{postnote}}
\DeclareCiteCommand{\footcite}[\mkbibfootnote]{\usebibmacro{prenote}}{\usebibmacro{citeindex}\printtext[bibhyperref]{ \usebibmacro{cite}}}{\multicitedelim}{\usebibmacro{postnote}}
\DeclareCiteCommand{\footcitetext}[\mkbibfootnotetext]{\usebibmacro{prenote}}{\usebibmacro{citeindex}\printtext[bibhyperref]{\usebibmacro{cite}}}{\multicitedelim}{\usebibmacro{postnote}}
\DeclareCiteCommand{\textcite}{\boolfalse{cbx:parens}}{\usebibmacro{citeindex}\printtext[bibhyperref]{\usebibmacro{textcite}}}{\ifbool{cbx:parens}{\bibcloseparen\global\boolfalse{cbx:parens}}{}\multicitedelim}{\usebibmacro{textcite:postnote}}

% header and footer settings
\pagestyle{scrheadings}
\clearscrheadfoot
\newcommand{\coursetitle}{Fancy course on foobar 2012/2013\xspace}
\lohead{\coursetitle}
\rohead{\headmark}
\setfootsepline{0.1pt}
\ofoot{\textsf{\thepage/\pageref{LastPageBeforeRefs}}}

% title page
\renewcommand*{\titlepagestyle}{empty}
\subject{\coursetitle}
\title{Work 01: On Barfoo coordinates}
\newcommand{\matrikelnummer}{08154711\xspace}
\author{Felix Smith}
\date{\today}
\newcommand{\short}{\blindtext}

% renders abstract along with declaration at titlepage with \publishers workaroung
\publishers{%
    \normalfont\normalsize%
    \vspace{2cm}
    \parbox{0.8\linewidth}{\textbf{Abstract}:~\short}\vfill
    \footnotesize
    \parbox{0.8\linewidth}{\textsf{\blindtext}}\vspace{0.4cm}
    \parbox{0.8\linewidth}{\textsf{Seattle, \underline{\hspace{2cm}} \hfill \underline{\hspace{6cm}}\newline place, date \hspace{3cm} student ID: \matrikelnummer \hfill signature}}
}

\begin{document}
\newgeometry{margin=0.1cm}
\maketitle\clearpage
\restoregeometry
\pagenumbering{arabic}
\tableofcontents\clearpage

% %%%%%%%%%%%%%%%%%%%%%%%%%%%%%%%%%%%%%%%%%%%%%%%%%%%%%%%%%%%%%%
% START YOUR DOCUMENT HERE
% %%%%%%%%%%%%%%%%%%%%%%%%%%%%%%%%%%%%%%%%%%%%%%%%%%%%%%%%%%%%%%

\section{Introduction}

% Оценка апроксимации для эффективной топографической симуляции ионно-лучевых процессов: 10К эВ аргон на кремний.

% Коротко:
%   Основоное предположение о существовании эффективной топографической симулиции состоит в том что распыление - это локальный процесс который зависит только от угла столкновения и не зависит от формы поверхности.Если учитывать *переотражение*, распылённые атомы *переотражаються* и не вызывают дальнейшего распыления когда они сталкиваються с другим учаством поверхности. Более того угловое распределение распылённых атомов определяется законом косинуса. Если учитываться ионное отражение, ионы не теряют энергию в процесе обратного рассеевания. Используя симуляцию бинарных столкновений (IMSIL) и сравнивая их с результатами полученными с помощью топографического симулятора (IonShaper®) мы видим что всем этим предположениям нужны уточнения для симуляции наноструктур прнебрегая распыления распылённых атомов. Кроме того мы видим что не локальные модели в сущности для ионно-лучевой наведённого облучения внутренней структуры.

% 1. Введение
% Ионные лучи это многостароннее и всё больше используемое стредства для создания наноструктур либо c помощью прямого фрезерования либо путём процессов в газовой среде. Настоящее обучение мотивированно усилиями чтобы создать штампы для *нано* литографии c массивными многоионными лазерными системами. Во всех приложениях симуляция может помочь понять физические процессы и оптимизировать форму структур.

% Большинство сущестувующих алгоритмов для численно эффективных топограыисеких симуляции ионно-лазерных процессов *track* поверхность проникновения явно, в соответсвуии с историей, точки поверхности считались соеденены прямыми сегментами.
% Все алгоритмы базируются на предположении что рассеивание- это функция угла между направлием ионов и локальной нормалью поверхности. Обычно это хорошая апрокимация для структур в диапозоне микрометров, но это становится под вопросом когда характерный размер порядка *the ion projected range* или ниже. Только некоторые алгоритмы включают *redeposition* и только несколько распыление отраженных ионов. Эти эффекты в основном важны в структурах с большим отношением сторон как в глубоких ямах. Рисунок 1 показывает контуры ям облучённых 200 нм широко гомогенным ионным лучём в два раличных момента времени, посчитанных с помощью програмного обеспечения IonShaper®. В любое время симуляция без *redeposition* и отражения (штрих-пунктирная линии) сравнима с симуляцией учитывающей только *redeposition* (пунктирная линия) и учитывающей как *redeposition* так и отражение (сплошная линия). Можно заметить что *redeposition* атомов рассееных из дна ямы ведёт к уменьшению ямы по направлению ко дну. На следующей стадии (не показанно), *redeposition* от граней ко дну  и к другой грани ведёт к уменьшению эфективности *фрезерования*. Отражение ионов от граней ведёт к формированию микроям на дне ямы возле граней. Более того, увеличение излучение на наклонёную поверхность микротрещин ведёт к дополнительной *redeposition* от граней.

% В подсчёте *redeposition* потока, закон косинуса используется чтобы описать угловое распределение распылённых ионов. Обратное рассевание предпологаеться без потери энергии. Далее, распылённые атомы которые достигают другую точку поверхности *redeposited* там и не вызывают рассевание. Эти предположения как и локальная аппрокимация рассеевания исследуются в этой статье используя симуляцию бинарных столкновений. Мы ограничиваем себя в 10к эВ Ar-ионы и Si-мишени т.к. это таш текуший фокус интересов.

% 2. Симуляция
% 2.1 Топографическая симуляция

% Двухмерная(2D) топограческие симуляции выполняются с IonShaper® программой. Коротко, поверхность описывается достаточным кол-вом точек которые движутся перпендикулярно к среднему наклону смежных сегментов. Скорость точек считается от потока атомов распылённых ионным лучом и от ионов отражённых от других частей поверхности и от потока *redeposited* атомово исходящих от распыления от остальных точек поверхности. Распыление считаеться локальным процессом, т.е. поток рыспылённых частиц в определённой точке пространства зависит только от потока падаюших ионов в эту точку и их угла относительной нормальни поверхности.

% В действительности, из-за ограниченных пределов отскока, распылённые атомы испускаються из облости вокруг точки падения. Чтобы исследовать этот эффект мы дополнительно реализовали не локальную модель потока расплыления. Кол-во отскоков в определённой точке поверхности (точке назначения) от столкновения ионов в другой точке поверхности (источнике) расчитывается от расстояния между двух точек и угла между прямой соединяющей эти точки и направлением падающих ионов.

% Изначальная версия IonShaper® включала модель вынужденного лучевого напыления которое включало посчёт предшествующей зоны наблюдения и простую реализацию не локального эффекта ионов/напылении. Было показанно что доля напыления пропоруианальна кол-ву отскоков достигших поверхность. Поэтому мы улучшили модель напыления используя модель основанную на отскоках опиисанную выше.

% 2.2 Бинарная симуляция столкновений.

% Бинарная симуляция столкновений сделанна с IMSIL алгоритмом. IMSIL было использованно для имплантационных исследований для одна и двух мерных целей. В этой работе мы только используем аморфные цели, т.к. Si просто аморфиризуется при ионом бомбондировании. IMSIL был улучшен для расчётов распыления в двух аспектах. Первое, реализована планарная модель поверхностного потенциала. Хотя это достаточно очивино для 1D целей, это более запцтанно в случае поверхнсти данной в полигонах в случае 2D из-за необходимости отслеживать трактории на некотором расстоянии вне цели и считать нормаль поверхности таю.Мы делает это путём охвата зоня симуляции правильной сеткой и считаем расстояние от каждой ячейки до поверхности. В процессе симуляции траектории отскоков расстояния отскоков от поверхности считается с помощью интерполяции в табличных величинах. Если расстояние от новой точки отскока из поверхности превышает максимум параметра pmax, тогда отскок возвращаеться в точку предидушего свободного полёта который есть расстояние pmax от поверхности. Нормаль поверхности в этой точке считается через градиент функции расстояния. В основном, отскоки покидающие поверхность проверяются путём помещения цели куда-нибудь ещё. В целях обучения, однако, отскоки останавливаются когда они покидают цель. Путём подгонки эксперементалього выхода распыления была определенна эффективная поверхнастная энергия связи в 4.1 eV.

% Второе, особое внимание должно быть уделенно чтобы обработать всплески столкновений правильно. Чтобы избижать нереалистичных близких столкновений когда помещаем цель, ионы должны начинать с расстояния pmax от поверхности. Только если внутри цели атомы цели настигаются тогда происходят столкновения. Однако, даже тогда результаты могут зависить от предположения о путях свободного пролёта. Это потому что распредделение пути свободного пролёта неявно определяет шероховатость поверхности. Поэтому мы используем распределение Пуассона для пути свободного пробега которое обеспечивает хорошую модель цели. Вместе с отклонением столкновений *partners* от цели это гарантирует постоянную атомную плотность внутри и нулевую плотность вне цели.

% 3 Результаты
% 3.1 Угловое распределение

% Рис.1 показывает угловое распределение распылёных и обратно рассеянных атомов полученное путём симуляции бинарных столкновений для углов наклона 0, 40, 70, 87.
% C увеличением угла наклона распределение возростающее отклоняется от закона косинуса (показанно на сфере). Отклонение вожно в основном при фрезерования глубоких ям т.к. поток в обратном направлении (направо Рис. 2) определяет сколько атомов могут покинуть яму.
% Угловое распределение обратно рассеянных электронов может быть только грубо описанно зеркальным отражением (прямык пунктирные линии). Распределение более смещанно по направлению *exit angles perpendicular to the surface* с большей стандартным отклонением в меньших углах наклона. Угловое распределение отраженных ионов определяет поверхность микроям показанных на Рис. 1 и поэтому должно быть описано точно.

% 3.2 Вторичное распыление
% IonShaper® предпологает что *redeposited* атомы не распыляются и что отражённые ионы распыляются с таким же выходом как и падающие ионы. Чтобы посчитать достоверность этого предположения мы посчитали средних выход распыления распылённых  и обратно рассеянных ионов для нормального наклона (Yss и Ysb соответственно) беря во внимание энергию распростронения распылённых ионов/обратно рассеяных ионов определённых симуляцией бинарных столкновений. Зависимость энергии нормально падающих распылённых ионов определенно с помощью аналитичной формулы. Т.к. в случае Yss мы заинтересованны только в грубой оценке, мы приближаем выход распылённых Si атомов распылением Ar ионов.
% Рис.3 показывает выход вторичного распыления Yss и Ysb. Для сильно *grazing* наклон Ysb близок к распылённию падающих ионов ( Ys = 1.47), но оно уменьшается значительно когда угол падения уменьшается (пример на 30% в 80; угол боковой стенки на Рис. 1 между 77 и 86). Для контраста, средний выход распыления Yss распылённых атомов скорее меньше и наверное незначительный в большенстве случаев.

% 3.3 Не локальные эффекты

% Чтобы исследовать нелокальные эффекты мы взяли контур на рис.1 в 4.67 с и посчитали основной поток распылённия в соответствии с разными моделями. Результаты полученные с локальной моделью показанны штрих-пунктирной линией на рис. 4, а результаты бинарного столкновения показанны сплошной линией. Можно заметить несколько удивительных отличий. Между 20нм и 50 нм по абсциссе локальные результаты переоценивают поток. В первом случае это потому что луч ограничен >20нм и вклад каскад отскоков столкновений ионов меньше 20нм пропушенно в результатах бинорного столкновия когда по иронии взято во внимание в локальной модели. В 50нм есть входящий поток больший 50нм, но точки падений глубже чем если контур был бы расширен больше 50нм с наклоном в <50нм. Поэтому сложнее порождать отскоки от ионов в >50нм чтобы вернуться к поверхности в <50нм. Это отраженно в результатах бинарного столкновения но только в локальной модели. В 30нм видно два четких пика в локальной модели которые (положительный пик). Эти изминения в наклоне настолько стремительны что они не отраженны в результатах бинарных столкновений из-за эффекта размытия конечного размера каскадов отскоков.

% Штриховая линия соответствует  IonShaper® результатом полученным с табличной моделью отскоков описанных вместе с *deposition* моделью. Так же можно заметить что они совпадают с результатами бинарных столкновений в большей части случаев. Пик не представленный в результате бинарных столкновений можеь быть замечен только на дне микроямы (>30нм). Это из-за тошло что путь от точки падения ион к точке назначения частично блокирован выгнутой поверхностю. Улучшенная модель описанная в конце секции 2.1 уменьшает этот пик, хотя некоторые отклонения от результатов бинарных столкновений остаются.

% Наконец, рис. 5 показывает IonShaper® результаты вынужденной ионно-лучевой  *deposition* в 25нм границей 10 keV Ar лучём. Штриховая линия соответствует локальной модели когда сплошная линия представляет не локальную модель включая изменения отскока. Так же нужно заметить что конечные размеры каскада отскоков значительно увеличивают ширину *deposited* столба. Этот эффект в основном выражен поскольку сравнительно низкая плотность отскоков вызывает значительный рост материала в *beam assisted deposition*.

% Выводы
% Наша оценка модели использованна для топографической симуляции ионой-лучевых процессов для 10keV Ar ионов и Si целей позволяет нам заключить следующее:

% (1) Закон косинуса это только грубая аппроксимация углового распределения распылённых атомов. В особенности, распыление глубоких ям требует точного описания распределения на больщих обратных уголах. Так же, зеркальное отражение это только грубая апроксимация для отражённых ионов. В обоих случаях реализация таблиц в топографическом симуляторе должно быть достачно хорошим.
% (2) Как было показынно на рис. 1 распыление отражённых ионов может быть важным эффектом. Распыление распылённых частиц похоже несущественно из-за не значительного средней энергии распылённых атомов.
% (3) Локальная модель распыления реализованная во всех топографических симуляторах на сегодня оказываеться достаточно неточна. Была предложенна модель основаная на пространственном распределении атомов на плоской поверхности и ожидается что она улучшает результаты на структурах малого размера. Нелокальная модель даже более важна для *ion beam assisted deposition*.

% Мы планируем расширить эту работу на другие энергии и типы ионов, изучать влияние предложенных моделей на форму поверхностей.

% Благодарности

% Эта работа было частично поддерженна Европейской коммисией путём финансирования проекта CHARPAN и Австрийским агенством продвижения, Австрийской нано инновационной программой, проектом NILaustria.

\subsection{Structure}

\texttt{section}, \texttt{subsection} und \texttt{subsubsection}. Try not to use \verb|\paragraph|.

\subsection{Figures}

\begin{figure}[h]
    \centering
    \includegraphics[width=4cm]{foobar} % Note that the extension is missing with purpose!
    \caption{This is a demo figure}
    \label{fig:demo1}
\end{figure}

\begin{figure}[h]
    \centering
    \subfloat[]{\includegraphics[width=1cm]{foobar}\label{subfig:demo2}}\hspace{0.2cm}
    \subfloat[]{\includegraphics[width=1cm]{foobar}\label{subfig:demo3}}\\
    \subfloat[]{\includegraphics[width=1cm]{foobar}\label{subfig:demo4}}\hspace{0.2cm}
    \subfloat[]{\includegraphics[width=1cm]{foobar}\label{subfig:demo5}}
    \caption{This are some other demo figures.}
    \label{fig:demo2-5}
\end{figure}

For more possibilities on figures see \texttt{l2picfaq.pdf}.

\subsection{Lists}

\begin{description}
    \item[Description:] Thats a description item

    \item[Itemize:] This is an itemization
        \begin{itemize}
            \item one
            \item two
            \item three
        \end{itemize}

    \item[Enumerate:] This is an enumeration
        \begin{enumerate}
            \item one
            \item two
            \item three
                \begin{enumerate}
                    \item A
                    \item B
                \end{enumerate}
        \end{enumerate}
\end{description}

\subsection{Tables}

\begin{table}[h]
    \centering
    \begin{tabular}{|r|l|}
        \hline
        7C0 & hexadecimal \\
        3700 & octal \\ \cline{2-2}
        11111000000 & binary \\
        \hline \hline
        1984 & decimal \\
        \hline
    \end{tabular}
    \caption{Taken from Wikibooks: \url{https://en.wikibooks.org/wiki/LaTeX/Tables}}
    \label{tab:demo1}
\end{table}

\subsection{Math}

\begin{equation}
    \int\limits_{-\infty}^{\infty}\! f(x,y+t)\circ g(x,z-t) dt
    \label{eq:demo1}
\end{equation}

\subsection{Literature references}

\blindtext \cite{Test2} foobar \ldots foobar \cite{Test1,Test4} barfoo \cite{Test3}.

\subsection{Document references}\label{subsec:references}

Figure~\ref{fig:demo1}, Figures~\ref{fig:demo2-5}, Tablle~\ref{tab:demo1}, Equation~\ref{eq:demo1} and Section~\ref{subsec:references}.

\subsubsection{Citation of URLs}

If neccessary to cite URLs please state the date and time of access as a note in the references.

\section{Getting started}

As software package we recommend \href{https://www.tug.org/texlive/}{TexLive}. Please use addition packages (\verb|\usepackage{foobar}| , etc.) with caution and very rarely.

\subsection{Tools}

\begin{itemize}
    \item \href{http://tex.stackexchange.com/questions/339/latex-editors-ides}{\LaTeX Editors/IDEs}
    \item \href{http://jabref.sourceforge.net/}{JabRef} for managing literature references
    \item \href{http://ctan.tug.org/tex-archive/support/latexmk/}{LatexMK} as build system
    \item \href{http://vim-latex.sourceforge.net/}{Vim-\LaTeX-suite} as extremly helpful \href{http://www.vim.org}{VIM} plugin
\end{itemize}

\section{Help}

Have a look on \href{https://en.wikibooks.org/wiki/LaTeX/}{Wikibook on \LaTeX}. See additional documents: \texttt{l2short.pdf}, \texttt{l2tabu.pdf}, \texttt{l2picfaq} as well as \url{tex.stackexchange.com} as a very helpful resource.

% %%%%%%%%%%%%%%%%%%%%%%%%%%%%%%%%%%%%%%%%%%%%%%%%%%%%%%%%%%%%%%
% HERE ENDS YOUR DOCUMENT
% %%%%%%%%%%%%%%%%%%%%%%%%%%%%%%%%%%%%%%%%%%%%%%%%%%%%%%%%%%%%%%

% Settings for the references (which are numbered in roman style)
\label{LastPageBeforeRefs}
\clearpage
\ofoot{\vspace{-0.1cm}\textbf{\large\thepage}/\textbf{\large\pageref{LastBibPage}}}
\pagenumbering{Roman}
\printbibliography
\label{LastBibPage}
\end{document}
