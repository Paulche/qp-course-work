%helloworld.tex
\documentclass{article}
\usepackage[warn]{mathtext}
\usepackage[utf8]{inputenc}
\usepackage{indentfirst}
\usepackage{amsmath}

\begin{document}
\begin{center}
     \textbf{Hello LaTeX!\dots}
\end{center}
\end{document}

%
% Оценка апроксимации для эффективной топографической симуляции ионно-лучевых процессов: 10К эВ аргон на кремний.

% Коротко:
%   Основоное предположение о существовании эффективной топографической симулиции состоит в том что распыление - это локальный процесс который зависит только от угла столкновения и не зависит от формы поверхности.Если учитывать *переотражение*, распылённые атомы *переотражаються* и не вызывают дальнейшего распыления когда они сталкиваються с другим учаством поверхности. Более того угловое распределение распылённых атомов определяется законом косинуса. Если учитываться ионное отражение, ионы не теряют энергию в процесе обратного рассеевания. Используя симуляцию бинарных столкновений (IMSIL) и сравнивая их с результатами полученными с помощью топографического симулятора (IonShaper®) мы видим что всем этим предположениям нужны уточнения для симуляции наноструктур прнебрегая распыления распылённых атомов. Кроме того мы видим что не локальные модели в сущности для ионно-лучевой наведённого облучения внутренней структуры.

% 1. Введение
% Ионные лучи это многостароннее и всё больше используемое стредства для создания наноструктур либо c помощью прямого фрезерования либо путём процессов в газовой среде. Настоящее обучение мотивированно усилиями чтобы создать штампы для *нано* литографии c массивными многоионными лазерными системами. Во всех приложениях симуляция может помочь понять физические процессы и оптимизировать форму структур.

% Большинство сущестувующих алгоритмов для численно эффективных топограыисеких симуляции ионно-лазерных процессов *track* поверхность проникновения явно, в соответсвуии с историей, точки поверхности считались соеденены прямыми сегментами.
% Все алгоритмы базируются на предположении что рассеивание- это функция угла между направлием ионов и локальной нормалью поверхности. Обычно это хорошая апрокимация для структур в диапозоне микрометров, но это становится под вопросом когда характерный размер порядка *the ion projected range* или ниже. Только некоторые алгоритмы включают *redeposition* и только несколько распыление отраженных ионов. Эти эффекты в основном важны в структурах с большим отношением сторон как в глубоких ямах. Рисунок 1 показывает контуры ям облучённых 200 нм широко гомогенным ионным лучём в два раличных момента времени, посчитанных с помощью програмного обеспечения IonShaper®. В любое время симуляция без *redeposition* и отражения (штрих-пунктирная линии) сравнима с симуляцией учитывающей только *redeposition* (пунктирная линия) и учитывающей как *redeposition* так и отражение (сплошная линия). Можно заметить что *redeposition* атомов рассееных из дна ямы ведёт к уменьшению ямы по направлению ко дну. На следующей стадии (не показанно), *redeposition* от граней ко дну  и к другой грани ведёт к уменьшению эфективности *фрезерования*. Отражение ионов от граней ведёт к формированию микроям на дне ямы возле граней. Более того, увеличение излучение на наклонёную поверхность микротрещин ведёт к дополнительной *redeposition* от граней.

% В подсчёте *redeposition* потока, закон косинуса используется чтобы описать угловое распределение распылённых ионов. Обратное рассевание предпологаеться без потери энергии. Далее, распылённые атомы которые достигают другую точку поверхности *redeposited* там и не вызывают рассевание. Эти предположения как и локальная аппрокимация рассеевания исследуются в этой статье используя симуляцию бинарных столкновений. Мы ограничиваем себя в 10к эВ Ar-ионы и Si-мишени т.к. это таш текуший фокус интересов.

% 2. Симуляция
% 2.1 Топографическая симуляция
% Двухмерная(2D) топограческие симуляции выполняются с IodShaper® программой. Коротко, поверхность описывается достаточным кол-вом точек которые движутся перпендикулярно к среднему наклону смежных сегментов. Скорость точек считается от потока атомов распылённых ионным лучом и от ионов отражённых от других частей поверхности и от потока *redeposited* частиц
%
%

